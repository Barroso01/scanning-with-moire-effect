%% NUESTRA PROPUESTA
%Advantages: More clarity by taking into account the four sections.
%Disadvantages: The object (al mover todo el experimento) manipulation may affect the results but this can be changed with a more technical setup.
%Posibles mejoras: buscar unwrappings mas eficientes. un setup profesional que no permita vibraciones ni variaciones en los resultados. Un fondo blanco que sea completamente plano.

%Alternative method for 360° scanning
%We tried to implement an alternative for the two mirror 360° \(buscar la referencia del dr rayas) scanning method by rotating the object and dividing it in four sections.
%Esto sería conclusión?

%% REPORTAR LAS POSIBLES APLICACIONES DENTRO DE LAS AREAS QUE MENCIONAMOS EN LA INTRODUCCION.

The three dimensional reconstruction with the double and quadruple matrix reconstruction proved to be a viable approach to obtain an accurate digital model of an object. Its simplicity lies in the scanning process which is straight forward and does not require major measurements to set up. By dividing the object into major sections and then scanning side sections individually the team was able to add the individual pieces into one complete figure as in a simple puzzle. \\

This method was developed as an alternative for other 360° scans like "3D shape and strain measurement of a thin-walled elastic cylinder using fringe projection profilometry" by Antonio de Jesús Ortiz-González, Amalia Martínez-García, Juan Benito Pascual-Francisco, Juan Antonio Rayas-Álvarez, and Alexis de Jesús Flores-García \cite{de20213d}. The target of the matrix reconstruction variations was to create an easier way to set the experiment up with less specialized materials while guaranteeing precise results. \\

The major issue with the matrix reconstruction variations comes from the need to guarantee that the object stays in the same position when rotated to minimize errors. This could be improved by finding a symmetrical axis in which the object could be rotated without being removed from the base. An important consideration from this method is to take into account the symmetry of the object to define which sides should be scanned and how many sections should the object be divided into. For the reconstruction itself, the main adjustments the team missed to implement due to the time constraints were the noise cleansing of the simulation and the spatial positioning of the different pieces. \\

An important remark the team realized with this experiment is that while using the matrix reconstruction method variations it is easier to obtain clearer images when using the least amount of pieces of the object while sacrificing detail. If the user wishes to obtain a clearer image then it is advised to use a greater number of images while having in mind that the clarity of the reconstruction might be compromised if not executed in an ideal way. As an additional note, the team would like to recommend using an even number of sections from the object to ease the simulation setup. \\

Further improvements include using a stress free background to reduce noise in the simulations. This can be achieved by using a sturdier material which is immune to deformities. The setup could also be further improved to be invulnerable to vibrations removing the fear of interacting with the objects conforming the setup. The wrapping method could be improved by using a Fourier method rather than equation \ref{eq:phase}. Furthermore, there may be more efficient unwrapping methods that could reduce the computational time and render better results. \\

This method could be implemented in a variety of fields from the medical imaging of superficial tissue or limbs to create adapted prostheses to the fashion industry when scanning body parts to create tailor made clothes and accessories fit for the client. In paleontology and archaeology a fringe scanning and matrix reconstruction method could allow researchers to create a model to analyze details of a desired object without fear of damaging the object by manipulating it. The possibilities to apply this method in different industries seem endless which makes improvements to this method necessary to be more adaptive to different circumstances.